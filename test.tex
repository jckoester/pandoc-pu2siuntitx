\documentclass{article}

\usepackage{siunitx}

\begin{document}

\section{Test}\label{test}

Testtext mit Pu inline \(\qty{50 }{\N\meter }\) und einmal ohne Dollars
\(\qty{50 }{\N\meter }\)

Physical Units (pu) (MathJax or KaTeX only, not for LaTeX)

\(\qty{123 }{\kJ }\) \(\qty{123 }{\mm\squared }\)

There are two conventions regarding the multiplication within units.

\(\qty{123 }{\J \s }\)

\(\qty{123 }{\J \cdot\s }\)

There are four conventions regarding divisions.

\(\qty{123 }{\kJ \per\mol  }\)

\(\qty{123 }{\kJ \per\mol  }\)

\(\qty{123 }{\kJ \per\mol  }\)

\(\qty{123 }{\kJ \per\mol  }\)

There are four main conventions for writing numbers in scientific
notation.

\(\qty{1.2e3 }{\kJ }\)

\(\qty{1,2e3 }{\kJ }\)

\(\qty{1.2E3 }{\kJ }\)

\(\qty{1,2E3 }{\kJ }\)

If you need more control than is offered here, take a look at the
siunitx extension.

\(\qty{13.3 }{\micro \meter }\) \(\qty{13.3 }{\mm }\)
\(\qty{.1 }{\mmol }\) \(\qty{2 }{\kA }\) \(\qty{3600 }{\kWh }\)
\(\qty{3600 }{\keV }\) \(\qty{25 }{\bequerel }\)

\end{document}